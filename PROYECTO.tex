% Formato de proyectos tipo FONDECyT.
% Documento adaptado de https://github.com/lfiguero/FondecyTeX por Roberto Navarro <roberto.navarro@udec.cl> 

% Este documento es modular, es decir, esta separado en varios
% archivos para facilitar su uso. Ver uso del paquete `subfiles`.
% https://www.overleaf.com/learn/latex/Multi-file_LaTeX_projects

\documentclass{article}
\usepackage[table]{xcolor}
\usepackage[paper=letterpaper,left=1.2cm,right=1.2cm,top=1cm,bottom=1cm,includefoot]{geometry}
\usepackage{fancyhdr}
\usepackage{tabularx}
\usepackage{multirow}
\usepackage{hyperref}
\usepackage{hhline}
\usepackage{amsmath}
\usepackage{enumitem}
\usepackage{pgfgantt}
\usepackage{graphicx}   \graphicspath{{img/}}

\usepackage[spanish]{babel}
\usepackage[utf8]{inputenc}


\usepackage{subfiles} % permite que el documento sea modular

%%% Comente/Descomente las siguientes lineas para cambiar la fuente del texto
\usepackage{DejaVuSans}
\renewcommand*\familydefault{\sfdefault}
% \usepackage{sansmath}
% \sansmath

\pagestyle{fancy}
\fancyhf{}
\renewcommand{\headrulewidth}{0pt}
\fancyfoot[L]{\textcolor[RGB]{127,127,127}{Proyecto de programación 2021}}

\definecolor{tcc}{RGB}{217,217,217} % Table cell color

\renewcommand\tabularxcolumn[1]{m{#1}}
\setlength{\arrayrulewidth}{0.5pt}
\renewcommand{\arraystretch}{2}

\renewcommand{\thesection}{\alph{section})}
\renewcommand{\thesubsection}{\alph{section}.\arabic{subsection}}

\renewcommand{\refname}{\vspace{-2ex}}

\begin{document}

\noindent
\begin{tabularx}{\textwidth}{|>{\columncolor{tcc}}m{5.7cm}|X|} \hline
  \textbf{Integrantes:} & \textbf{Nombres separados por coma}
  \\ \hline
  \textbf{Título del proyecto:} & \textbf{Ingrese un título representativo del producto final esperado}
  \\ \hline
  \textbf{Asignatura:} & \textbf{Física Computacional II (510240)} \newline Departamento de Física \newline Facultad de Ciencias Físicas y Matemáticas \newline Universidad de Concepción
  \\ \hline
\end{tabularx}

%%% Aqui se incluyen los archivos externos
\bigskip \subfile{resumen}
\newpage \subfile{propuesta}
\newpage
\include{bib}
\newpage
\include{justification}
\newpage
\include{resources}

\end{document}
