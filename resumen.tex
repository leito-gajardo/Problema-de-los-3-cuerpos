\documentclass[proyecto.tex]{subfiles}

\begin{document}

\noindent\textbf{RESUMEN:}

\bigskip \noindent \par Nuestro proyecto emerge del reconocido problema que surgió de la Ley de Gravitación Universal de Newton: \textbf{el problema de los 3 cuerpos}, el cual consiste en encontrar las ecuaciones que describen 3 cuerpos sometidos a una atracción gravitatoria mutua planteadas las condiciones iniciales de velocidad y posición, en otras palabras, se traduce en la \textbf{resolución de ecuaciones diferenciales}. Este problema, como ya se sabe, tiene un alto nivel de complejidad, ya que como se ha visto en algunas simulaciones computacionales asociadas, como las 695 familias de trayectorias determinadas el año 2017 por XiaoMing Li y ShiJun Liao, se observan evidentes comportamientos caóticos en los movimientos de los tres cuerpos. Pero ¿Por qué sólo 695 familias y con resolución computacional?, ¿Por qué mejor no encontrar todas las trayectorias posibles para cualquier condición inicial de forma analítica? Resulta ser, que este es un problema irresoluble para los matemáticos, por la cantidad de ecuaciones diferenciales y la cantidad de incógnitas que se encuentran en el camino de la resolución. Pero a su vez, es un problema sustancial para infinidad de aplicaciones en áreas de astronomía, vuelos espaciales, dinámica galáctica, formación estelar y la determinación de trayectorias para misiones de naves espaciales, por lo que resulta ser incluso necesario al momento de abordar estas aplicaciones.

 Podemos encontrar aplicaciones más básicas de este problema que involucran satélites artificiales, donde se puede plantear \textbf{el problema restringido de los tres cuerpos}, donde una de las masas es mucho menor que las otras dos, por lo que se puede despreciar su influencia gravitacional hacia los objetos mas masivos. Justamente, este problema restringido, al ser más simple, ha sido el más práctico para las aplicaciones pertinentes, tales como: sistema Tierra-Luna-Nave espacial, Tierra-Sol-Luna o Sol-Júpiter-Asteroide.

 Entonces, ya comprendido el problema a abordar y sus importantes aplicaciones, podemos aclarar el objetivo de nuestro proyecto: Nos proponemos, como objetivo principal, resolver computacionalmente algún(os) caso(s) particular(es) del problema de los tres cuerpos, usando métodos numéricos vistos en el curso \textit{física computacional I} (si es necesario, se usarán métodos externos), adjuntando a su vez sus respectivas simulaciones de las trayectorias. Para ello, se realizarán en primera instancia cálculos analíticos y numéricos para \textbf{el problema de los 2 cuerpos}, para luego proceder con los mismos cálculos para el problema de los 3 cuerpos. Se utilizarán métodos de resoluciones analíticas para ecuaciones diferenciales y para las resoluciones numéricas métodos como el de Runge-Kutta, el de Euler, e incluso, de ser necesario, métodos de ceros de funciones o de integración de funciones. Por otro lado, para las simulaciones, se considerará la posibilidad de “salirse” del apartado de Python, y buscar otros lenguajes que podrían resultar más simples al momento de simular en 2D o 3D, en particular, se evaluará el uso de MATLAB.
 
 %\textbf{Número máximo de palabras en esta sección: 500 palabras}.

\bigskip



\end{document}
